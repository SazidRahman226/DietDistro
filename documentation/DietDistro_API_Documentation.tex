\documentclass[11pt, a4paper]{article}

% --- Essential Packages ---
\usepackage[utf8]{inputenc} % For character encoding
\usepackage{geometry}       % For page margins
\geometry{a4paper, margin=1in}
\usepackage{xcolor}         % For coloring text
\usepackage{hyperref}       % For clickable links and TOC
\usepackage{listings}       % For code highlighting
\usepackage{array}          % For advanced table features
\usepackage{tabularx}       % For tables that automatically size columns

% Explicitly load the language definitions
% OR a more common way to ensure it's loaded if not automatically
% \lstset{language=json} % Just by setting it, it tries to load

% --- Hyperref Setup ---
\hypersetup{
    colorlinks=true,
    linkcolor=blue,
    filecolor=magenta,
    urlcolor=cyan,
    pdftitle={My API Documentation},
    pdfauthor={Your Name},
    pdfsubject={API Reference},
    pdfkeywords={API, Documentation, LaTeX, REST}
}

% --- Listings Styles ---
% Define styles for different code types
% JSON Style

\lstdefinestyle{jsonstyle}{
    language=json,
    basicstyle=\ttfamily\small,
    breaklines=true,
    showstringspaces=false,
    commentstyle=\color{gray!70},
    stringstyle=\color{teal},
    numberstyle=\tiny\color{gray},
    keywordstyle=\color{blue}, % JSON "keywords" are usually null, true, false
    columns=fixed,
    numbers=left,
    xleftmargin=1em,
    frame=single,
    frameround=trbl,
    framesep=5pt,
    rulecolor=\color{gray!30}
}

% HTTP Request Style (can use general text, but keywords might be useful)
\lstdefinestyle{httpstyle}{
    language=bash, % Use bash for shell commands like cURL or generic for HTTP
    basicstyle=\ttfamily\small,
    breaklines=true,
    commentstyle=\color{green!50!black},
    keywordstyle=\color{red!70!black},
    stringstyle=\color{purple},
    numbers=none,
    columns=fixed,
    xleftmargin=1em,
    frame=single,
    frameround=trbl,
    framesep=5pt,
    rulecolor=\color{gray!30},
    % Custom keywords for HTTP methods/headers
    morekeywords={GET, POST, PUT, DELETE, Host, Content-Type, Authorization, Bearer}
}

% General Code (e.g., Python/JavaScript)
\lstdefinestyle{codestyle}{
    language=Python, % Or JavaScript, C, etc.
    basicstyle=\ttfamily\small,
    breaklines=true,
    commentstyle=\color{green!50!black},
    keywordstyle=\color{blue},
    stringstyle=\color{red!70!black},
    numbers=left,
    columns=fixed,
    xleftmargin=1em,
    frame=single,
    frameround=trbl,
    framesep=5pt,
    rulecolor=\color{gray!30}
}


% --- Document Start ---
\begin{document}

\title{Diet Distro}
\author{Team Almost N308}
\date{\today}
\maketitle

% --- Table of Contents ---
\tableofcontents
\newpage

% --- Section: Introduction ---
\section{Introduction}
Welcome to the documentation for Diet Distro. This API allows you to manage users, health profile, diet planning, diet sharing through a set of RESTful endpoints. All requests and responses are typically in JSON format.

% --- Section: Authentication ---
\section{Authentication APIs}
\subsection{\textbf{\large Register a New User}}
\begin{itemize}
    \item \textbf{POST} \colorbox{gray!20}{\tiny /api/auth/register}
    \item \textbf{Request Body:}
    \begin{lstlisting}[style=jsonstyle]
{
    "name": "Md. Sazidur Rahman",
    "email" : "rahmansazid2@gmail.com",
    "password" : "password",
    "age" : 25,
    "gender" : male,
    "height" : 1.7,
    "weight" : 70
}
\end{lstlisting}
    \item \textbf{Response:} \colorbox{gray!20}{201 created}
    \begin{lstlisting}[style = jsonstyle]
{
    "id": 1,
    "name": "Md. Sazidur Rahman",
    "email": "rahmansazid2@gmail.com",
    "created_at": "2024-07-29T08:45:00Z"
}
    \end{lstlisting}
    
\end{itemize}

\subsection{\textbf{\large Login}}
\begin{itemize}
    \item \textbf{POST} \colorbox{gray!20}{\tiny /api/auth/login}
    \item \textbf{Request Body:}
        \begin{lstlisting}[style=jsonstyle]
{
    "email" : "rahmansazid2@gmail.com",
    "password" : "password"
}
        \end{lstlisting}
        OR
        \begin{lstlisting}[style=jsonstyle]
{
    "name" : "Md . Sazidur Rahman ",
    "password" : "password"
}
        \end{lstlisting}
    \item \textbf{Response:}  \colorbox{gray!20}{200 OK}
        \begin{lstlisting}[style=jsonstyle]
{
    "token" : "<JWT_TOKEN>"
    "expires_in": <time in sec>
}
        \end{lstlisting}
\end{itemize}


% --- Section: User Management ---
\section{User Management}
This section details the endpoints related to user management.

\subsection{Get User Details}
\label{api:getUserDetails}

This endpoint retrieves the details of a single user by their ID.

\subsubsection*{Endpoint}
\begin{lstlisting}[style=httpstyle, numbers=none]
GET /api/users/{id}
\end{lstlisting}
OR\begin{lstlisting}[style=httpstyle, numbers=none]
GET /api/users/{user_name}
\end{lstlisting}
OR
\begin{lstlisting}[style=httpstyle, numbers=none]
GET /api/users/{user_email}
\end{lstlisting}

\subsubsection*{Response Codes}
\begin{tabularx}{\linewidth}{|l|l|X|}
    \hline
    \textbf{Code} & \textbf{Status} & \textbf{Description} \\
    \hline
    \texttt{200} & \texttt{OK} & User details retrieved successfully. \\
    \hline
    \texttt{404} & \texttt{Not Found} & The specified user ID does not exist. \\
    \hline
    \texttt{401} & \texttt{Unauthorized} & Invalid or missing authentication token. \\
    \hline
\end{tabularx}

\subsubsection*{Successful Response (200 OK)}
\begin{lstlisting}[style=jsonstyle]
{
    "id": 123,
    "name": "Md. Sazidur Rahman",
    "email" : "rahmansazid2@gmail.com",
    "age" : 25,
    "gender" : male,
    "height" : 1.7,
    "weight" : 70
    "created_at": "2024-01-15T10:00:00Z"
}
\end{lstlisting}

\subsubsection*{Error Response (404 Not Found)}
\begin{lstlisting}[style=jsonstyle]
{
  "error": "User not found",
  "code": 404
}
\end{lstlisting}

% --- Cross-referencing example ---
For details on retrieving user information, refer to Section \ref{api:getUserDetails}.

\end{document}